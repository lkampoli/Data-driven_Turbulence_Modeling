\chapter{Clustering analysis}
% https://cran.r-project.org/web/packages/ClusterR/vignettes/the_clusterR_package.html
% https://cran.r-project.org/web/packages/sBIC/vignettes/GaussianMixtures.pdf
% https://towardsdatascience.com/mixture-modelling-from-scratch-in-r-5ab7bfc83eef
% https://www.biorxiv.org/content/10.1101/2019.12.20.884551v2.full
% https://www.r-bloggers.com/2017/02/an-intro-to-gaussian-mixture-modeling/
% https://tinyheero.github.io/2015/10/13/mixture-model.html
%
Cluster analysis or clustering is the task of grouping a set of objects in such a way that objects in the same group (called a cluster) are more similar (in some sense or another) to each other than to those in other groups (clusters). It is the main task of exploratory data mining, and a common technique for statistical data analysis, used in many fields, including machine learning, pattern recognition, image analysis, information retrieval, bioinformatics, data compression, and computer graphics.

The most prominent examples of clustering algorithms are Connectivity-based clustering (hierarchical clustering), Centroid-based clustering (k-means, k-medoids, ...), Distribution-based clustering (Gaussian mixture models) and Density-based clustering (DBSCAN, OPTICS, ...).

The \textbf{ClusterR} package consists of centroid-based (k-means, mini-batch-kmeans, k-medoids) and distribution-based (GMM) clustering algorithms. Furthermore, the package offers functions to:
\begin{itemize}
    \item    validate the output using the true labels,
    \item     plot the results using either a silhouette or a 2-dimensional plot,
    \item     predict new observations,
    \item     estimate the optimal number of clusters for each algorithm separately.
\end{itemize}

The Defined distance (DBSCAN) algorithm finds clusters of points that are in close proximity based on a specified search distance. The Self-adjusting (HDBSCAN) algorithm finds clusters of points similar to DBSCAN but uses varying distances, allowing for clusters with varying densities based on cluster probability (or stability). The Multi-scale (OPTICS) algorithm orders the input points based on the smallest distance to the next point. A reachability plot is then constructed, and clusters are obtained based on the fewest points to be considered a cluster, a search distance, and characteristics of the reachability plot (such as the slope and height of peaks).
%
\url{https://geographicdata.science/book/notebooks/10_clustering_and_regionalization.html}\\
\url{https://www.monolithai.com/blog/machine-learning-for-turbulence-modeling}\\